\section*{چرکنویس}
\subsection*{تحریک}




\underline{
این قسمت باید تحت عنوان تحریک مغز قرار بگیرد
}

تحریک مغز به عنوان یک درمان کمکی از اختلالات عصبی مختلف با موفقیت متغیر تحت تحقیق قرار گرفته است.

یک چالش ، دانش محدود در مورد اهداف عصبی موثر برای یک مداخله ، همراه با دانش محدود در مورد مکانیسم های عصبی تحریک مغز است.

از یک سو با انگیزه شواهد اخیر مبنی بر اینکه فعالیت های نوسانی در سیستم های عصبی در تنظیم عملکردها و اختلالات عملکرد مغز ، به ویژه موارد اختلالات عصبی ویژه بیماران مسن ، نقش دارند ،
و از طرف دیگر که ممکن است از تکنیک های تحریک مغز برای تعامل با این نوسانات مغزی به روشی کنترل شده استفاده شود ، ما در اینجا پتانسیل تعدیل نوسانات مغز را به عنوان یک استراتژی موثر برای مداخلات بالینی تحریک مغز بررسی می کنیم.

ما در ابتدا شواهد مربوط به پروفایل های نوسانی غیرطبیعی را که با طیف وسیعی از اختلالات عصبی در افراد مرتبط است (به عنوان مثال ، بیماری پارکینسون ، بیماری آلزایمر، سکته مغزی ، صرع) بررسی می کنیم.
سپس ، ما سیگنال های فعالیت غیر طبیعی شبکه را بررسی می کنیم تا با درمان نرمال شود و یا پیش بینی پیشرفت بیماری یا بهبودی آن باشد.
سپس این سوال را می پرسیم که پروتکل های  تحریک مغز موجود تا چه اندازه متناسب با این نوسانات و احتمالاً اختلالات عملکردی تنظیم شده اند.

با این حال ، تاکنون ، درک نقش این ریتم ها در رفتار انسان و بروز علائم در بیماری ها محدود است.


هزاران نورون فعالیت خود را همگام می کنند تا الگوی نوسانی نوعی تولید کنند که می تواند از طریق الکتروانسفالوگرام از الکترودهای روی پوست سر یا از طریق پتانسیل های میدان محلی  یا ضبط داخل جمجمه از الکترودهای کاشته شده در مغز اندازه گیری شود.

تکنیک های تحریک مغز با القای و تعدیل فعالیت نوسانی مداوم قادر به ایجاد تغییرات عملکردی در مغز هستند.بنابراین ، این تکنیک ها ممکن است نقش بالقوه ای در تشخیص و درمان اختلالات عصبی ، آشکار کردن پاسخ نوسانی غیرطبیعی و یا استفاده در تلاش برای برقراری تعادل مجدد فعالیت در مدارهای عصبی غیرطبیعی داشته باشند.


اتصالات درون مغز پویا هستند، که با افزایش سن تغییر می کند و می تواند با آموزش های شناختی و بدنی تعدیل شود.
گروههای مختلفی از سلولهای عصبی تمایل دارند که فعالیت خود را در فرکانسهای خاصی که به صورت ریتم تعریف شده اند همزمان کنند ، که در پنج باند فرکانسی طبقه بندی شده اند: دلتا (<4 هرتز) ، تتا (4-8 هرتز) ، آلفا (8-12 هرتز) ، بتا (12–30 هرتز) و گاما (30–90 هرتز). فرکانس های بالاتر (> 90 هرتز) به عنوان فعالیت نوسانی فرکانس بالا در نظر گرفته شده اند

\lr{ EEG} به راحتی می تواند نوسانات از باند دلتا تا بتا را تشخیص دهد ، در حالی که برای ضبط فعالیت های گاما و فعالیت نوسانی فرکانس بالا کمتر مناسب است ، زیرا شدت سیگنال این فعالیت ها در \lr{ EEG} از فعالیت فراوان (عمدتا عضله و جریان مستقیم اطراف) پوست سر خارج نمی شود.تأثیر محدود فعالیت عضلانی در \lr{MEG}و ضبط داخل جمجمه ، این دو روش را برای تحلیل نوسانات فرکانس بسیار بالا نیز مناسب می کند. در بخش های بعدی ، ما ادبیات موجود را به عنوان شواهدی در مورد نقش فیزیولوژیکی و پاتوفیزیولوژیکی این ریتم های مختلف بررسی خواهیم کرد.

شواهد متعددی نشان داده اند که فرکانس و یا فاز فعالیت نوسانی موضعی و همچنین فعالیت نوسانی بین نواحی مختلف از درگیری عملیات شناختی مختلف خبر می دهد که بر بروز رفتارهای خاص تاثیر میگذارند.
علاوه بر این ، نقایص عصبی مربوط به عملکردهای حرکتی ، دیداری-فضایی و حافظه با خرابی فعالیت نوسانی عصبی که در باندهای فرکانسی خاص در داخل و در سراسر نواحی مغز و قشر مغز فعالیت می کند ، مرتبط شده اند.

این گزارش ها علاقه مندی روزافزونی را برای دستکاری ریتم های مغزی انسان برای درک بهتر نقش علیت فیزیولوژیکی آنها در کدگذاری عملیات شناختی و درمان برخی از علائم اختلالات عصبی روانپزشکی همراه با اختلالات همزمان سازی عصبی ایجاد کرده است.

رویکردهای غیر تهاجمی تحریک مغزی ، مانند تحریک مغناطیسی یا تحریک جریان مستقیم، برای تقریباً دو دهه برای دستکاری و تعدیل فعالیت مغز استفاده شده است ، که نویدبخش نوعی درمان برای  بیماری های عصبی روانی است

اعمال این تکنیک ها با پارامترها و پیکربندی های تحریک مختلف (به عنوان مثال، تنظیم فرکانس یا فاصله بین برست های مختلف برای تحریک مغناطیسی) توانایی این را دارد تا تحریک پذیری سلول های عصبی را در محدوده مشخصی از مغز به صورت انلاین (همزمان) یا افلاین (طولانی مدت) تغییر دهد. همچنین این اثرات با دنبال کردن مسیر های اتصال درون مغز می تواند در میان شبکه های مغزی منتشر شوند. 


%\subsection{دلتا}
%در شرایط فیزیولوژیکی ، فعالیت دلتا برجسته ترین ویژگی  \lr{ EEG }در خواب حرکت غیر سریع چشم
%\LTRfootnote{ Non-rapid eye movement sleep (NREM) }
%انسان است.
%منشا آن نورونهای قشر مغز است و به عنوان یک واسطه احتمالی از انعطاف پذیری سیناپسی وابسته به خواب ارائه شده است که با همگام سازی حالت تحریک پذیری گروه های عظیمی از سلولهای عصبی قشر مغز فرایندهای حافظه کورتیکو-هیپوکامپ را تسهیل می کند.
%هنگام بیداری ، فعالیت دلتا در شرایط فیزیولوژیکی تقریباً وجود ندارد ،
%\\
%\lr{but it appears both after subcortical brain lesion sparing cerebral cortex and after the induction of cortical plasticity}

%\subsection{ الکتروانسفالوگرافی}
%نورون ها در شرایط خاصی تکانه الکتریکی کوچکی ایجاد می کنند؛ این تکانه ها اساس روش های ثبت الکتروفیزیولوژی هستند. گرچه هر نورون به تنهایی جریان خیلی کوچکی تولید می کند، قشر مغز از تعداد زیادی نورون تشکیل شده است؛ بنابراین اگر بسیاری از این نورون ها همزمان کار مشابهی انجام دهند، جریان های تولید شده به اندازه کافی بزرگ می شود، به طوری که از راه پوست سر می تواند ثبت گردد. این جریان ها باید بزرگ باشند؛ زیرا الکترودهای ثبت کننده روی پوست سر از منبع تکانه های الکتریکی با پرده های اطراف مغز، فضای مایع مغزی-نخایی، استخوان، چربی و پوست جدا شده است. به این ترتیب فعالیت ثبت شده در هر الکترود، به فعالیت کلی هزاران نورون وابسته است.
\subsection*{نوشته های جسته و گریخته که شاید درون متن بدرد بخوره}
ارتباطات نورونی وابسته است به مولفه های آناتومیک که نورون های منفرد را به هم وصل می کند (ساختاری) و فرآیند انتقال اطلاعات (عملکرد)؛ هر دو جنبه روی کارایی نهایی دستگاه عصبی اثر کی گذارند.


\lr{mean field models of neural populations under electrical stimulation-2020-plos computational biology:}

تحریک الکتریکی سیستم عصبی ابزاری کلیدی برای درک و مطالعه دینامیک شبکه های عصبی و در نهایت برای توسعه درمان بالینی است. بسیاری از کاربردهای تحریک الکتریکی جمعیت زیادی از سلول های عصبی را تحت تاثیر قرار می دهد و شبیه سازی و تحلیل شبکه های بزرگ نورونی کار سخت و دشواری است. ما در این مطالعه یک مدل تقلیل یافته میدان متوسط از نورون های مهاری و تحریکی 
\lr{AdExIF} 
را بررسی کرده ایم که می تواند در مطالعه اثرات تحریک الکتریکی روی جمعیت نورونی بزرگ استفاده شود. 

ورودی های الکتریکی ضعیف که به مغز در شرایط آزمایشگاهی (
\lr{in vivo} 
)
به کمک تحریک الکتریکی فراجمجمه ای یا در قشر جدا شده در شرایط آزمایشگاهی (
\lr{in vitro}
)
می تواند روی رفتار دینامیکی آن جمعیت نورونی تاثیر بگذارد. با این حال ساز و کارهای دقیقی که فعالیت کل جمعیت نورونی را کنترل و تنظیم می کند و اینکه چرا واکنش به تحریک در آزمایش ها بسیار متنوع هستند، هنوز به درستی درک نشده اند. علی رغم ناشناخته بودن جنبه های مختلف، تکنیک های تحریک الکتریکی برای درمان بیماری های عصبی در انسان ها در حال توسعه هستند. برای درک هر چه بهتر این برهمکنش ها، در بیشتر اوقات لازم است که شبکه بزرگی از نورون ها شبیه سازی و تحلیل شوند که از نظر محاسباتی بسیار دشوار است و نیاز به تلاش فراوان دارد. ما در این پژوهش یک مدل تقلیل یافته از جمعیت های نورونی جفت شده ارائه کرده ایم که نماینده تکه ای از بافت قشر مغز است. ما نشان دادیم که میدان های الکتریکی که اغلب در آزمایش های تحریک مغز بکار برده می شوند می تواند منجر به تقلید
\LTRfootnote{entrainment}
 نوسانات عصبی در سطح جمعیت شود.

اختلال وارد کردن به سامانه به منظور کشف خواص دینامیکی آن پارادایمی شناخته شده است که در علوم فیزیکی موفقیت آن ثابت شده است. این روش برای مقیاس های مختلفی که در سیستم عصبی مورد مطالعه قرار می گیرد نیز موفقیت های در خور توجهی داشته است. در مطالعات متعددی نشان داده شده است که روش های تحریک غیر تهاجمی مغز در شرایط آزمایشگاهی (
\lr{in vivo} 
)
مثل تحریک الکتریکی فراجمجمه ای با جریان متناوب (
\lr{tACS}
)
فعالیت نوسانی و همچنین عملکرد مغز را تغییر می دهد و روش های جدیدی را برای درمان اختلالات بالینی مثل صرع یا برای بهبود تثبیت حافظه در هنگام خواب فراهم کرده است. با این حال پژوهشگران هنوز به درک کاملی از چگونگی تاثیر تحریک الکتریکی بر شبکه های بزرگ نورونی دست نیافتند. به همین دلیل ما یک چارچوب محاسباتی برای مطالعه برهمکنش ورودی های الکتریکی متغیر با زمان با رفتارهای دینامیکی جمعیت های نورونی بزرگ ارایه کرده ایم.


\subsection*{چند سیناپسی با تاخیر}
این معادله معروف که معادله تحول ولتاژ غشا نورون هست، بدون تاخیر زمانی 
\begin{equation}
    \tau_m \frac{dv_i}{dt}=E_l-v_i-\sum_{j=1}^{N} r_m g_{ij}S_{ij}(v_i-E_{syn,j})
\end{equation}
این اندیس
$j$
در جمله 
$E_{syn,j}$
هم بخاطر این در نظر بگیریم که بعضی از نورون ها تحریکی و بعضی ها مهاری هستن که با وجود این اندیس
$j$
میشه بعدن شبکه ای درست کرد مخلوط از نورون های تحریکی و مهاری.

و این معادله که معادله تحول احتمال باز بودن  دریچه های یونی در سیناپس هست
\begin{equation}
    \tau_s \frac{dS_{ij}}{dt}=-S_{ij}
\end{equation}

\begin{figure}
    \centering
        \begin{tikzpicture}
        \Vertices{fig/tikzNet/3neuron1ver.csv}
        \Edges{fig/tikzNet/3neuron1edg.csv}
        \end{tikzpicture}
    \caption{نمای شبکه}
    \label{fig:3neuron1}
\end{figure}
برای مثال اگر قرار باشه همه معادلات مربوط به شبکه نشان داده شده در شکل 
\ref{fig:3neuron1}
را بنویسیم خواهیم داشت:


%\begin{equation}
    \begin{align}
        & \tau_m \frac{dv_1}{dt}=E_l-v_1 + R_m I_{ext} -r_m  g_{12}S_{12}(v_1-E_{syn,2}) -r_m  g_{13}S_{13}(v_1-E_{syn,3}) \\
        & \tau_m \frac{dv_2}{dt}=E_l-v_2+ R_m I_{ext} -r_m g_{21}S_{21}(v_2-E_{syn,1}) -r_m  g_{23}S_{23}(v_2-E_{syn,3}) \\
        & \tau_m \frac{dv_3}{dt}=E_l-v_3+ R_m I_{ext} -r_m g_{31}S_{31}(v_3-E_{syn,1}) -r_m  g_{32}S_{32}(v_3-E_{syn,2}) \\
        & \tau_s \frac{dS_{12}}{dt}=-S_{12} \\
        & \tau_s \frac{dS_{21}}{dt}=-S_{21} \\
        & \tau_s \frac{dS_{13}}{dt}=-S_{13} \\
        & \tau_s \frac{dS_{31}}{dt}=-S_{31} \\
        & \tau_s \frac{dS_{23}}{dt}=-S_{23} \\
        & \tau_s \frac{dS_{32}}{dt}=-S_{32} \\
    \end{align}
%\end{equation}

اگر قرار باشه تاخیر رو وارد کنیم معادله دیفرانسیل به شکل زیر تغییر می کنه
\begin{equation}
    \tau_m \frac{dv_i}{dt}=E_l-v_i-\sum_{j=1}^{N} r_m g_{ij}S{ij}(t-\tau_{ij})(v_i-E_{syn,j})
\end{equation}
اما معادله دوم(همان معادله تحول احتمال باز بودن دریچه های یونی در سیناپس) بدون تغییر باقی می مونه

\begin{figure}
    \centering
        \begin{tikzpicture}
        \Vertices{fig/tikzNet/2neurondelay1ver.csv}
        \Edges{fig/tikzNet/2neurondelay1edg.csv}
        \end{tikzpicture}
    \caption{نمای شبکه چند سیناپسی با تاخیر}
    \label{fig:2neurondelay1}
\end{figure}

برای مثال اگر قرار باشه معادلات مربوط به شبکه نشان داده شده در شکل 
\ref{fig:2neurondelay1}
را بنویسیم خواهیم داشت

\begin{equation}
    \begin{align*}
         \tau_m \frac{dv_1}{dt} & =E_l-v_1 + R_m I_{ext} \\
         \tau_m \frac{dv_2}{dt} & =E_l-v_2+ R_m I_{ext} -r_m g_{21}S_{21}(t-\tau_{21})(v_2-E_{syn,1}) \\ 
        & -r_m  g_{21'}S_{21'}(t-\tau_{21'})(v_2-E_{syn,1}) \\
         \tau_s \frac{dS_{21}}{dt} & =-S_{21} \\
    \end{align*}
\end{equation}


% \begin{figure}
%     \centering
%         \begin{tikzpicture}
%         \Vertex{A} \Vertex[x=2]{B}
%         \Edge[Direct](A)(B)
%         \end{tikzpicture}
%     \caption{کپشن}
%     \label{fig:my_label}
% \end{figure}

% \begin{figure}
%     \centering
%         \begin{tikzpicture}
%         \Vertices{vertices.csv}
%         \Edges{edges.csv}
%         \end{tikzpicture}
%     \caption{نمای شبکه}
%     \label{fig:my_label}
% \end{figure}

