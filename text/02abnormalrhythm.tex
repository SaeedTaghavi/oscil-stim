
\section{ ریتم های بهم ریخته در بیماری ها}

\subsection{پارکینسون}
بیماری پارکینسون  ، که با لرزش ، سفتی و کندی حرکت مشخص می شود ، یکی از شایع ترین اختلالات تخریب عصبی در جهان است. مشخصه پاتولوژیک بیماری پارکینسون از دست دادن سلولهای دوپامینرژیک در جسم سیاه
\LTRfootnote{Substantia nigra}
و سایر مناطق مغز است.
مکانیسم‌هایی که باعث از بین رفتن سلول‌های دوپامینرژیک و منجر به بروز نشانه‌های حرکتی بیماری پارکینسون می‌شود هنوز به طور کامل مشخص نشده است. مجموعه‌ای از شواهد در حال رشد، نوسانات عصبی غیر طبیعی درون و بین نواحی مختلف مغز در بیماران پارکینسونی نشان می دهد. 
%پژوهش های متعددی نشان داده اند که بیماری پارکینسون با افزایش فعالیت نوسانی مغز در باند بتا همراه است.

%الگوهای نوسانی منحصر به فردی با ناهنجاری‌های حرکتی خاص در بیماری پارکینسون همراه هستند.
 روش‌های درمانی مانند استفاده از داروهای دوپامینرژیک و تحریک عمقی مغز که این الگوهای نوسانی عصبی غیرطبیعی را مختل می‌کند، باعث بهبود علائم در بیماران علامت می‌شود.

مطالعه فعالیت نورونی هم در انسان و هم در مدل های جانوری بیماری پارکینسون شواهدی در مورد افزایش فعالیت نوسانی در باند بتا در هسته های 
\lr{GPe}
، 
\lr{Gpi} 
و 
\lr{STN}
ارائه کرده است. جالب است که ضایعات یا تحریک هسته های 
\lr{Gpi} 
 و یا
 \lr{STN}
در درمان علائم بیماری پارکینسون موثر هستند، احتمالا به این دلیل که باعث کاهش یا از بین رفتن همگامی غیر طبیعی در خروجی بیزل گنگلیا می شود
\cite{chen2011stimulation, moro2002impact, jenkinson2011new, hammond2007pathological}
.
\subsection{صرع}
صرع رایج ترین اختلال مغزی جدی با شیوع تقریبا کمتر از ۱ درصد کل جمعیت عمومی است.
صرع دربرگیرنده حمله‌های برگشت‌پذیر است که در آن ها فعالیت مغزی با تشدید شدن یا همزمانی گسترده فعالیت های نورون های قشری
\LTRfootnote{cortical neurons}
مختل می شود. 
حمله ها همیشه در نتیجه همزمانی و تشدید نابهنجار و فعالیت ناگهانی یک یا چند توده از نورون‌های‌ قشری رخ می‌دهد. این فعالیت نابهنجار معمولا به روش ثبت موج نمای الکتریکی مغز
\LTRfootnote{EEG}
قابل مشاهده و ثبت است. همزمانی فعالیت سلول‌های عصبی پتانسیل ریتمیک و نسبتا بزرگی ایجاد می‌کند. گرچه بسیاری از افراد مبتلا به صرع، حمله‌های خود را با داروهای در دسترس به شکل موثری مهار می‌کنند، اما گروه کوچکی از این بیماران ممکن است به روش درمانی دیگر، مانند جراحی یا تحریک مغز نیاز داشته باشند.

تحریک عصبی واگ
\LTRfootnote{Vagus Nerve Stimulation (VNS)}
یک تکنیک برای درمان صرع است و یکی از اقدامات نیمه تهاجمی در کنترل تشنج‌ها می‌باشد که در طولانی مدت موجب بهبود کنترل تشنج‌ها می‌گردد. با این روش درمانی تشنج‌ها کمتر و کوتاه‌تر می‌شود و نیاز به درمان دارویی کاهش می‌یابد و در برخی موارد تشنج بیماران بطور کامل از بین می‌رود.

%\subsection{اختلال دوقطبی}
%\subsection{اختلال اوتیسم}

\section{درمان}
برای همه این بیماری‌ها، با توجه به سطح بیماری و پاسخ بیمار به درمان های مختلف، سطوح مختلف درمانی وجود دارد؛ اما در حالت کلی همه درمان‌ها را می توان به سه گونه مختلف تقسیم کرد: درمان‌های رفتاری، درمان‌های شیمیایی، و درمان‌های فیزیکی.

%\begin{itemize}
%  \item درمان‌های رفتاری
%  \item درمان‌های شیمیایی
%  \item درمان‌های فیزیکی
%\end{itemize}

درمان های رفتاری تلاشی برای آموزشِ تنظیم کارکرد درست مغز توسط خود مغز می‌باشد، برای مثال روشی مثل نوروفیدبک
\footnote{بازخورد عصبی یا نوروفیدبک در اصل، نوعی بیوفیدبک است که با استفاده از ثبت امواج الکتریکی مغز و دادن بازخورد به فرد تلاش می‌کند که نوعی خودتنظیمی را به فرد آموزش دهد}
.
درمان های شیمیایی در همه موارد همراه با مصرف دارو یا مواد شیمیایی است که روی کل دستگاه عصبی و یا بخشی از آن اثر می‌گذارد. در درمان‌های فیزیکی اساسا اثرگذاری روی دستگاه عصبی از طریق مکانیزم‌های فیزیکی مختلف اتفاق می‌افتد. تحریک مغز که در این دسته از درمان‌ها قرار دارد به صورت مستقیم روی نرخ فعالیت نورون‌ها تاثیر می گذارد. تمرکز ما بیشتر روی دسته سوم درمان ها  به طور خاص روش های مختلف تحریک مغز و اثرات آن روی شبکه های نورونی خواهد بود.


تحریک مغز به روش های مختلفی انجام می‌گیرد که دسته بندی‌های متفاوتی نیز دارد. برای مثال، از نظر جنس محرک استفاده شده می توان تحریک الکتریکی، تحریک مغناطیسی، تحریک اپتیکی، و تحریک مکانیکی-صوتی را نام برد.
%
%\begin{itemize}
%  \item تحریک الکتریکی
%  \item تحریک مغناطیسی
%  \item تحریک اپتیکی
%  \item تحریک مکانیکی-صوتی
%\end{itemize}

%\subsection{درمان های رفتاری}
%\subsection{درمان های شیمیایی}
%\subsection{اختلال درمان های فیزیکی}
