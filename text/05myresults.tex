\section{نتایج }
\subsection{تضعیف وابسته به فاز نوسانات غیر عادی در سیستم عصبی}
همان‌طور که تا کنون اشاره شد بیماری‌های مختلفی هستند که با افزایش غیر طبیعی نوسانات مغزی همراه هستند. 
 یکی از راه‌های درمانی برای مقابله با این شرایط استفاده از تحریک عمیق مغز می‌باشد. تحریک عمیق مغز به وسیله الکترودهایی که درون بافت مغز قرار می‌گیرند، اعمال می‌شود.

\begin{figure}
	\centering
	\includegraphics[width=0.9\textwidth]{dbs-protocols}
    \caption{
    پروتکل‌های مختلف در اعمال تحریک عمیق مغز.
در شکل بالایی تحریک عمیق مغز حلقه-باز را می‌بینیم؛ در این پروتکل پارامترهای تحریک ثابت هستند. 
در شکل میانی، بر اساس پایش و نظارت تنها یک پارامتر از سیگنال ورودی (در اینجا توان سیگنال ورودی)، خروجی دستگاه نوسان‌ساز را تعیین می‌کند. در پروتکل نشان داده شده در شکل میانی، تنها در صورتی که توان امواج بتا از آستانه مشخصی بیش‌تر شود، تحریک هایی با دامنه و فرکانس ثابت به مغز اعمال می‌شود. اما در شکل پایین بیش از یک مشخصه از سیگنال ورودی روی تعیین نحوه و مشخصات اعمال تحریک اثر می‌گذارد. مثلا در همین شکل، قدرت تحریک نیز بر اساس قدرت سیگنال ورودی تنظیم می‌شود.
    }
    \label{fig:dbs-protocols}
\end{figure}


پس از کاشت الکترود، برای تضعیف نشانه‌های پاتولوژیک و ایجاد اثرات جانبی کمتر باید پارامترهای تحریک نظیر فرکانس تحریک و شکل موج تحریک تنظیم شوند. به کمک مدل‌های محاسباتی می‌توان پارامترهای بهینه تحریک عمیق مغز را تخمین زد. در طول سال‌های گذشته پروتکل‌های مختلفی برای نحوه اعمال تحریک‌ها ارایه شده است که برخی از آن‌ها در شکل
\ref{fig:dbs-protocols}
نمایش داده شده‌اند. 

پروتکل‌هایی که در آن‌ها پارامترهای تحریک ثابت باشند تحریک عمیق مغز حلقه-باز نامیده می‌شوند.
 از منظر فرکانسی، تحریک عمیق مغز حلقه-باز دو دسته دارد. اگر فرکانس اعمال تحریک از ۱۰۰ هرتز کمتر باشد، فرآیند تحریک در دسته تحریک عمیق مغز فرکانس پایین جای می‌گیرد و اگر فرکانس تحریک بیشتر از ۱۰۰ هرتز باشد فرآیند تحریک در دسته تحریک عمیق مغز با فرکانس بالا قرار می‌گیرد. اگر تحریک عمیق مغز با پارامتر‌های غیر ثابت، واکنشی و براساس فعالیت در حال وقوع در مغز اعمال شود، تحریک عمیق مغز حلقه-بسته نامیده می‌شود. تحریک حلقه-بسته را می‌توان بر اساس  پایش و نظارت تعداد متفاوتی از نشانگرهای مختلف اعمال کرد (شکل
 \ref{fig:dbs-protocols}
 و
  \ref{fig:dbs-closed-loop})
  . این رهیافت می‌تواند اثرات جانبی را کاهش داده و اثربخشی درمان را افزایش می‌دهد.
 
\begin{figure}[t!]
     \centering
     \begin{subfigure}[t]{0.45\textwidth}
         \centering
         \includegraphics[width=\textwidth]{dbs-closedloop1}
         \caption{
 اعمال تحریک بر اساس دامنه فعالیت نوسانی در حال انجام در مغز.
         }
%         \label{fig:tdcs}
     \end{subfigure}
     \
     %\hfill
     \begin{subfigure}[t]{0.45\textwidth}
         \centering
         \includegraphics[width=\textwidth]{dbs-closedloop2}
         \caption{
 اعمال تحریک بر اساس فاز فعالیت نوسانی در حال انجام در مغز.
         }
%         \label{fig:tacs}
     \end{subfigure}
        \caption{
دو پروتکل معروف در رهیافت حلقه-بسته
         }
        \label{fig:dbs-closed-loop}
\end{figure}



علاوه بر استراتژی کلی تحریک، پارامترهای شکل موج نیز از پارامترهای مهم در تحریک عمیق مغز می‌باشد. برخی از گروه‌های پژوهشی به بررسی نحوه اثر و پیدا کردن پارامترهای بهینه شکل موج می‌پردازند. تعریف دقیق پارامترهای شکل موج در تحریک عمقی مغز می‌تواند از آسیب دیدن بافت مغز یا الکترود جلوگیری کند، به کنترل بهتر فعالیت عصبی کمک کند و هزینه انرژی را کاهش دهد که باعث افزایش عمر باتری می‌شود و از این رو از جراحی‌های جایگزینی دستگاه جلوگیری می‌شود. در شکل 
\ref{fig:dbs-pulse-shape}
می‌توانید نمونه‌هایی از شکل موج‌های مختلف را مشاهده کنید. 


\begin{figure}[b!]
     \centering
     \begin{subfigure}[t]{0.3\textwidth}
         \centering
         \includegraphics[width=\textwidth]{pulse-shape1}
%         \caption{الکترودهای تحریک عمیق مغز کاشته شده در سر }
%         \label{fig:tdcs}
     \end{subfigure}
     \
     %\hfill
     \begin{subfigure}[t]{0.3\textwidth}
         \centering
         \includegraphics[width=\textwidth]{pulse-shape2}
%         \caption{الکترودهای تحریک عمیق مغز کاشته شده در سر }
%         \label{fig:tacs}
     \end{subfigure}
     \
  %   \hfill
     \begin{subfigure}[t]{0.3\textwidth}
         \centering
         \includegraphics[width=\textwidth]{pulse-shape3}
%         \caption{    نوسان‌سازهای کار گذاشته شده درون قفسه سینه}
%         \label{fig:trns}
     \end{subfigure}
        \caption{
شکل موج‌های مختلف در تحریک عمیق مغز که بین قسمت‌های آنُدیک و کاتُدیک تاخیر متفاوتی دارند.
         }
        \label{fig:dbs-pulse-shape}
\end{figure}



موفقیت تحریک عمیق مغز حلقه-بسته به طراحی استراتژی تحریک بستگی دارد که بر اساس آن نوسانات عصبی تضعیف می‌شوند.  ساختن مدلی ریاضی است که بتواند پاسخ نوسانات عصبی به تحریک در حالت‌های مختلف مغز را توصیف کند، گامی مهم در راستای طراحی و پیشنهاد استراتژی‌های بهتر است.
ما در بررسی‌های اولیه، جمعیت نورونی که نوسانات پاتولوژیک تولید می‌کند را با شبکه‌ای از نوسانگرهای جفت شده مدل‌سازی کردیم. معادله تحول این نوسانگرها بر اساس مدل کوراموتو نوشته شده‌است. همچنین تحریک‌ها در یک لحظه خاص وقتی که فاز سیستم مقدار مشخصی است به اندازه تابع پاسخ فاز هر نوسانگر به آن‌ها وارد می‌شود.
رفتار وابسته به زمان مدل کوراموتو به صورت عددی در یک شبکه کامل مورد مطالعه قرار گرفته است. ما رفتار ناهمگام شدن را با حل این معادلات روی شبکه کامل بررسی کردیم.
\begin{equation}
    \frac{d \theta_l}{dt} = \omega_l + \frac{k}{N} \sum_{j=1}^{N} \sin(\theta_j -\theta_l) + I \delta(t-t_{stim}) Z(\theta_l)
    \label{eq:kuramotoModelres}
\end{equation}
جمله اول، 
$\omega_l$
فرکانس طبیعی نوسانگر 
$l$ 
ام می باشد.
جمله دوم توصیف کننده برهمکنش بین نوسانگرهای مختلف است که 
$k$
ضریب جفت شدگی می‌باشد و قدرت جفت شدگی بین هر جفت نوسانگر و در نتیجه تمایل نوسانگرها برای همگامی را کنترل می کند. 
جمله سوم اثر تحریک را توصیف می کند؛ شدت تحریک توسط 
$I$
مشخص شده است و تابع 
$\delta(t-t_{stim})$
در همه زمان ها صفر است به جز در زمان هایی که تحریک اعمال می‌شود که برابر است با یک.
تابع پاسخ فاز یک نوسانگر منفرد نیز
$Z(\theta_l)$
است.


\begin{figure}
	\centering
	\includegraphics[width=0.45\textwidth]{kur1000orderParam}
    \caption{
نمودار اندازه پارامتر نظم بر حسب ضریب جفت شدگی. هر نقطه (برای هر مقدار ضریب جفت شدگی) از متوسط گیری پارامتر نظم ۴ سیستم با شرایط اولیه متفاوت تولید شده است. برای شبیه سازی هر سیستم از تعداد ۱۰۰۰ نوسانگر کوراموتو استفاده شده است. فرکانس ذاتی نوسانگرها به صورت تصادفی از توزیع نرمال (
$\mu=1.0$  و $\sigma=0.1$
) و فاز اولیه نوسانگرها نیز به صورت تصادفی از توزیع یکنواخت در بازه 
$[0,2\pi)$
انتخاب شده است. انتگرال‌گیری برای ۱۰۰۰ گام زمانی به طول
 $dt \approx 0.01$
 به روش اویلر انجام شده است که ۱۰ گام نهایی برای محاسبه پارامتر نظم آن سیستم میانگین گیری شده اند. محدوده مشخص شده در 
 $k \approx 0.4$
 محدوده مورد نطر ماست زیرا نوسانگرها به طور کامل همگام نمی‌شوند.
    }
    \label{fig:orderParam-k}
\end{figure}

همان‌طور که پیش از این اشاره کردیم، اندازه پارامتر نظم معیاری از همگامی است؛ که مقادیر صفر و یک آن به ترتیب با حالت کاملا ناهمگام و حالت کاملا همگام متناظر هستند. وقتی تحریکی به سیستم وارد نشود، نوسانگرها تمایل به همگامی دارند. تمایل به همگامی توسط ضریب جفت شدگی کنترل می شود؛ به طوری که هرچه ضریب جفت شدگی بزرگ‌تر باشد همگامی سریع‌تر اتفاق میفتد. با توجه به اینکه ما در نظر داریم تحریک‌ را بر اساس فاز نوسانگرها اعمال کنیم، پس ضریب جفت شدگی را طوری انتخاب می‌کنیم که نوسانگرها کاملا همگام نشوند. بنابراین، با توجه به شکل 
\ref{fig:orderParam-k}
مقدار ضریب جفت‌شدگی 
$k \approx 0.4$
را انتخاب می‌کنیم. در نهایت، بر اساس معادله تحول 
\ref{eq:kuramotoModelres}
در زمان‌های مختلف سیستم را تحریک می‌کنیم.

در شکل 
\ref{fig:kur-compare-thetas}
تحول زمانی فازهای نوسانگرها در سه حالت مختف قابل مشاهده است؛ در شکل بالایی هیچ تحریکی به سیستم وارد نشده است، در شکل میانی تحریک در لحظه 
$t=55.7$
و در نمودار پایینی در لحظه 
$t=59.3$
اعمال شده است. 

\begin{figure}
	\centering
	\includegraphics[width=0.9\textwidth]{kur1000-compare-thetas}
    \caption{
نمودار تحول فاز نوسانگرها در سه حالت مختلف. نمودار بالایی بدون اعمال تحریک. در نمودار میانی تحریک در لحظه 
$t=55.7$
و در نمودار پایینی در لحظه 
$t=59.3$
اعمال شده است.
    }
    \label{fig:kur-compare-thetas}
\end{figure}

در شکل‌های 
\ref{fig:stim557}
و
\ref{fig:stim593}
جزییات بیشتری از اثر تحریک بر فعالیت نوسانگرها مشخص است. همانطور که در شکل \ref{fig:stim557} مشاهده می‌کنید، تحریک باعث پخش و ناهمگام شدن فاز نوسانگرها می‌شود؛ که این مسئله بوضوح در نمودار فراوانی فاز نوسانگرها قبل و بعد از اعمال تحریک پدیدار می‌شود. علاوه بر این‌ها در نمودار تحول زمانی پارامتر نظم نیز مشاهده می‌شود که در لحظه اعمال تحریک پارامتر نظم افت پیدا می‌کند اما در ادامه به حالت بدون تحریک میل می‌کند.

\begin{figure}
     \centering
     \begin{subfigure}[t]{0.3\textwidth}
         \centering
         \includegraphics[width=\textwidth]{k1000stim-orderParam1}
%         \caption{الکترودهای تحریک عمیق مغز کاشته شده در سر }
%         \label{fig:tdcs}
     \end{subfigure}
     \
     %\hfill
     \begin{subfigure}[t]{0.2\textwidth}
         \centering
         \includegraphics[width=\textwidth]{distro1}
%         \caption{الکترودهای تحریک عمیق مغز کاشته شده در سر }
%         \label{fig:tacs}
     \end{subfigure}
     \
  %   \hfill
     \begin{subfigure}[t]{0.4\textwidth}
         \centering
         \includegraphics[width=\textwidth]{k1000theta-stim1}
%         \caption{    نوسان‌سازهای کار گذاشته شده درون قفسه سینه}
%         \label{fig:trns}
     \end{subfigure}
        \caption{
اعمال تحریک در لحظه 
$t=55.7$.
در شکل سمت چپ نمودار تحول زمانی فاز نوسانگرها را مشاهده می‌کنیم. در شکی میانی توزیع فاز در لحظه های قبل (آبی) و بعد (قرمز) از اعمال تحریک را می‌بینیم و در شکل سمت راست تحول زمانی پارامتر نظم برای سیستم تحریک شده (آبی) و سیستم تحریک نشده  (سبز).
         }
        \label{fig:stim557}
\end{figure}

در شکل \ref{fig:stim593} مشاهده می‌بینید تحریک باعث جمع و همگام‌تر شدن فاز نوسانگرها می‌شود؛ که این مسئله بوضوح در نمودار فراوانی فاز نوسانگرها قبل و بعد از اعمال تحریک قابل مشاهده است. علاوه بر این‌ها در نمودار تحول زمانی پارامتر نظم نیز مشاهده می‌شود که در لحظه اعمال تحریک پارامتر نظم صعود قابل توجهی دارد اما در ادامه به حالت بدون تحریک میل می‌کند.

\begin{figure}
     \centering
     \begin{subfigure}[t]{0.3\textwidth}
         \centering
         \includegraphics[width=\textwidth]{k1000stim-orderParam2}
%         \caption{ تحول زمانی پارامتر نظم }
%         \label{fig:tdcs}
     \end{subfigure}
     \
     %\hfill
     \begin{subfigure}[t]{0.2\textwidth}
         \centering
         \includegraphics[width=\textwidth]{distro2}
%         \caption{فراوانی فاز نوسانگرها قبل و بعد از تحریک }
%         \label{fig:tacs}
     \end{subfigure}
     \
  %   \hfill
     \begin{subfigure}[t]{0.4\textwidth}
         \centering
         \includegraphics[width=\textwidth]{k1000theta-stim2}
%         \caption{تحول زمانی فاز نوسانگرها }
%         \label{fig:trns}
     \end{subfigure}
        \caption{
اعمال تحریک در لحظه 
$t=59.3$.
در شکل سمت چپ نمودار تحول زمانی فاز نوسانگرها را مشاهده می‌کنیم. در شکی میانی توزیع فاز در لحظه های قبل (آبی) و بعد (قرمز) از اعمال تحریک را می‌بینیم و در شکل سمت راست تحول زمانی پارامتر نظم برای سیستم تحریک شده (نارنجی) و سیستم تحریک نشده  (سبز).
         }
        \label{fig:stim593}
\end{figure}

اگر بیشینه تغییرات پارامتر نظم برای هر سیستم تحریک شده نسبت به سیستم تحریک نشده را 
$\Delta r$
بنامیم و نمودار تغییرات آن بر حسب زمان اعمال تحریک را رسم کنیم شکل 
\ref{fig:orderparam-stim}
را بدست می‌آوریم.





\begin{figure}
     \centering
     \begin{subfigure}[t]{0.4\textwidth}
         \centering
         \includegraphics[width=\textwidth]{k1000stim-orderParam}
         \caption{
بیشینه تغییرات پارامتر نظم سیستم تحریک شده و سیستم تحریک نشده بر حسب زمان اعمال تحریک.
}
         \label{fig:orderparam-stim}
     \end{subfigure}
     \
     %\hfill
     \begin{subfigure}[t]{0.4\textwidth}
         \centering
         \includegraphics[width=\textwidth]{orderParam-prc}
         \caption{
نمودار بیشینه تغییرات پارامتر نظم بین سیستم تحریک شده و تحریک نشده (خط نقطه قرمز) و تابع پاسخ فاز جمعیت (آبی).
 }
%         \label{fig:tacs}
     \end{subfigure}

        \caption{
نمودار بیشینه تغییرات پارامتر نظم به همراه نمودار پاسخ فاز جمعیت نوسانگرها
         }
%        \label{fig:stim593}
\end{figure}

\begin{figure}[h!]
	\centering
	\includegraphics[width=0.5\textwidth]{multiple-stim}
    \caption{
با اعمال تحریک‌های پیاپی بر اساس فاز سیستم در دوره‌های مختلف می‌توان پارامتر نظم را کاهش داد و همچنین حالت دینامیکی سیستم را نیز تغییر داد.
    }
%    \label{fig:kur-compare-thetas}
\end{figure}


%
%\begin{figure}
%	\centering
%	\includegraphics[width=0.5\textwidth]{k1000stim-orderParam}
%    \caption{
%بیشینه تغییرات پارامتر نظم سیستم تحریک شده و سیستم تحریک نشده بر حسب زمان اعمال تحریک.
%    }
%    \label{fig:orderparam-stim}
%\end{figure}



%
%\begin{figure}
%	\centering
%	\includegraphics[width=0.7\textwidth]{theta-dist}
%    \caption{
%اعمال تحریک در لحظه 
%$t=61.9$.
%    }
%%    \label{fig:kur-compare-thetas}
%\end{figure}
%

%
%\begin{figure}
%	\centering
%	\includegraphics[width=0.6\textwidth]{orderParam-prc}
%    \caption{
%نمودار بیشینه تغییرات پارامتر نظم بین سیستم تحریک شده و تحریک نشده (خط نقطه قرمز) و تابع پاسخ فاز جمعیت (آبی).
%    }
%%    \label{fig:kur-compare-thetas}
%\end{figure}

این مدل سازی‌ها توصیف خوبی از نحوه تغییر نوسانات مغزی بعد از اعمال تحریک ارایه می‌دهند و پیشنهاد می‌کنند که دامنه نوساناتی که نشانه بیماری هستند را می‌توان با اعمال تحریک در فازهای مشخصی از نوسانات، اصلاح کرد. همچنین، این نتایج، نشان می‌دهد که با توجه به منحنی پاسخ فاز نوسانگرها می‌توان بهترین زمان اعمال پالس را برای بیشترین کاهش در دامنه نوسان پیش‌بینی کرد.



بعلاوه با اعمال تحریک‌های پیاپی بر اساس فاز سیستم در دوره‌های مختلف می‌توان پارامتر نظم را کاهش و حالت دینامیکی سیستم را تغییر داد.
\\ \\ \\ \\

%نوسانات همگام درون و بین نواحی مغز به پردازش عادی اطلاعات در مغز کمک می‌کنند؛ اما در برخی از بیماری‌های سیستم عصبی، این نوسانات در مقایسه با مغز سالم تقویت می‌شوند. به عنوان مثال می‌توان به نوسانات تقویت شده بتا (با فرکانس تقریبی ۲۰ هرتز) که بین قشر و هسته زیرتالاموسی در بیماران پارکینسونی ثبت شده است، اشاره کرد. تحریکِ عمیقِ مغز، درمانی مؤثر برای اختلالات عصبی متنوعی از جمله پارکینسون و لرزش اساسی است. در حال حاضر در روش تحریک عمیق مغز، قطاری از پالس‌های الکتریکی با بسامد ثابت توسط الکترود‌هایی که در مغز کاشته شده‌اند به محل‌های هدف در مغز اعمال می شوند.  تحریک عمیق مغز حلقه‌بسته یک رهیافت جدید و  امید‌بخش است که تحریک بر اساس وضعیت بیمار اعمال می‌شود. در این رهیافت جدید به جای آنکه از قطاری از پالس‌های با بسامد ثابت استفاده کنند، تحریک بر اساس فعالیت مغزی بیمار اعمال می‌شود.  این روش توانایی زیادی برای پیشرفت در بهره‌وری، ثمربخشی و کاهش اثرات جانبی دارد. بهبود این روش وابسته به ابداع کردن یک راهبرد تحریک است که نوسانات فعالیت‌های عصبی که نشانه بیماری هستند را کاهش دهد. یکی از راه‌های رسیدن به این هدف مدل سازی‌های نظری و محاسباتی است. این مدل سازی‌ها توصیف خوبی از نحوه تغییر نوسانات مغزی بعد از اعمال تحریک ارایه می‌دهند. برای مثال، مدل سازی های محاسباتی پیشنهاد می کنند که دامنه نوساناتی که نشانه بیماری هستند را می توان با اعمال تحریک در فازهای مشخصی از نوسانات، اصلاح کرد. در این مطالعه، در نواحی‌ای که نوسانات نشانه بیماری را تولید می‌کنند به جای نورون‌ها از نوسانگر های جفت شده استفاده می‌کنیم و نشان می دهیم که اگر در فاز و دامنه مشخصی تحریک را اعمال کنیم، سیستم عصبی چطور به آن تحریک پاسخ خواهد داد. همچنین نشان می‌دهیم که با توجه به منحنی پاسخ فاز نوسانگرها می‌توان بهترین زمان اعمال پالس را برای بیشترین کاهش در دامنه نوسان پیش‌بینی کرد.


%\breakpage