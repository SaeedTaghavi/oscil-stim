\begin{abstract}
طی سه دهه اخیر مطالعات و آزمایشات گسترده انجام شده نشان داده‌اند که ریتم‌های مغزی و نوسانات گروهی سلول‌های عصبی نقشی بنیادی در کنترل تبادل اطلاعات بین نواحی مختلف مغز و در نتیجه در پردازش اطلاعات و شکل‌دهی کارکردهای شناختی مختلف از جمله ادراک حسی، توجه، حافظه، و آگاهی ایفا می‌کنند. اختلال در الگوی نوسانی نواحی مختلف مغز در بسیاری از بیماری‌های سیستم عصبی مشاهده و ثبت شده است. امروزه شواهد محکمی وجود دارند که اختلال در الگوی نوسان، نه تنها یک نشانه بلکه دلیل بوجود آمدن ناهنجاری‌های شناختی و عصبی است. بنابر این شواهد، بازتولید الگوی نوسانی مغز سالم در درمان‌های مختلف موجب بهبود نشانه‌های بیماری می‌شود و تحقق آن به کمک روش‌های درمانی مختلف از جمله درمان‌های دارویی، بازتوانی شناختی و تحریک مغزی از جمله معیارهای مورد نظر پژوهشگران است. 

تحریک مغزی یکی از این روش‌های درمانی است که همزمان با درمان دارویی و یا در مواردی که مداخلات دارویی پاسخ مناسبی نمی‌دهد، استفاده شده و کارآیی آن در طیف وسیعی از بیماری‌های شناختی و عصبی مشاهده شده است. با این حال سازوکار تاثیر آن بر شبکه‌های عصبی و علت کارایی آن در بسیاری از موارد ناشناخته است و کاربرد و توسعه آن بر اساس تجربیات مستند شده یا آزمون و خطا انجام می‌گیرد.

برای تحلیل نوع تاثیر تحریک مغزی بر فعالیت جمعی نورون‌ها، شناخت مواردی چون اصول حاکم بر دینامیک فعالیت شبکه‌های عصبی، خواص الکتریکی نورون‌ها، خواص فیزیکی و ساختار شبکه مورد تحریک  لازم است. درک صحیح نظری از رفتار دینامیکی جمعیت‌های نورونی تحت تاثیر تحریک می‌تواند به بهبود روش‌های تحریک کمک کند و همچنین پیش‌بینی اثر تحریک بر روی عملکرد مغز را امکان‌پذیر نماید. در پژوهش پیش رو اثر تحریک بر فعالیت نوسانی جمعی نورون‌ها و اثر آن در بیماری‌ها و ناهنجاری‌های مختلف بررسی خواهد شد.

کلمات کلیدی: 
همگامی سیستم عصبی، تحریک مغز، نوسانات عصبی

\newpage


\end{abstract}